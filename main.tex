\documentclass[lang=cn,newtx,10pt,scheme=chinese]{elegantbook}

\title{C 语言课程讲义}

\author{LMXZ}
\date{2025-10-21}
\version{0.1}

\setcounter{tocdepth}{3}

\cover{cover.jpg}

% 本文档命令
\usepackage{array}
\usepackage{titlesec}

\input{lmbook.tex}

% 修改标题页的橙色带
\definecolor{customcolor}{RGB}{32,178,170}
\colorlet{coverlinecolor}{customcolor}
\usepackage{cprotect}

\addbibresource[location=local]{reference.bib} % 参考文献,不要删除

\begin{document}

\maketitle
\frontmatter

\tableofcontents

\mainmatter

\chapter{快速入门}

\section{快速配置开发工具}

前期建议使用易配置的 Code::Blocks 作为编程工具。可以在这里下载:

\href{https://www.codeblocks.org/downloads/binaries/}
{https://www.codeblocks.org/downloads/binaries/}

选择“Windows(64 bit)”(如果你用的是 Windows 的话)下的 \lstinline{codeblocks-X.Xmingw-setup.exe}。这个版本自带了 GCC 编译器,安装后即可使用。

使用 Code::Blocks 只是为了快速上手,但并不推荐使用作为长期的开发工具,等你熟悉了 C 语言之后,可以考虑使用更强大的 IDE,比如 Visual Studio Code、CLion 等。

\section{第一个程序,Hello World!}

Hello World 是指输出“Hello, World”的简单程序,在开始学习一门语言时,一般会从这里开始入手。

\begin{lstlisting}[language=C]
    #include <stdio.h>
    
    int main() {
        printf("Hello, World!"); // 输出 Hello World
        return 0; // 程序结束
    }
\end{lstlisting}        

现在程序写好了,需要运行时就点击上面的“\includegraphics{image/btn_compile_run.png}”按钮。

目前你需要了解的是:

\begin{itemize}
    \item \lstinline{#include <stdio.h>} 是预处理指令,用于包含标准输入输出库,提供 \lstinline{printf} 函数的定义;
    \item \lstinline{int main()} 是主函数的定义,程序从这里开始执行;
    \item \lstinline{printf("Hello, World!");} 用于输出内容到屏幕,括号内的内容是要输出的字符串;
    \item \lstinline{return 0;} 表示程序正常结束,返回值为 0;
    \item “\lstinline{// 输出 Hello World}” 这种字样是注释,用于做批注,以便人类能看懂;计算机并不会理会“\lstinline{//}”以及后面的内容;
    \item 在 \lstinline{int main} 里面的每一行的结尾都必须有一个分号 “\lstinline{;}”;
    \item 输入程序的时候要特别注意,把输入法关掉,否则你输入的符号会变成全角字符;程序中的符号应该统一为半角字符。
\end{itemize}

\section{输入输出}

刚才的例子中,\lstinline{printf("XX")} 是用于输出内容的。
然而一个程序往往需要输入、计算、输出三步。我们看第二个例子:

\label{sec:input-output}
\begin{lstlisting}[language=C]
    #include <stdio.h>
    
    int main() {
        int a, b; // 定义变量 a b;
        printf("Please input the value of a and b:");
        scanf("%d%d", &a, &b); // 读入 a b
        int c = a + b; // 计算 a+b 的结果,存到变量 c 里面
        printf("The result of %d + %d is: %d", a, b, c); // 输出结果
        return 0;
    }
\end{lstlisting}

这个程序会读取用户输入的两个数字,相加并输出。
运行程序,输入“\lstinline{3 5}”(注意中间有空格),你会看到程序输出了“\lstinline{The result of 3 + 5 is: 8}”。

这里你需要记住三句话的用法:

\begin{itemize}
    \item \lstinline{int} 用于定义整数型变量,一行可以定义多个变量;定义时如果在后面加上 “\lstinline{= xx}”,就会同时给它赋值。
    \item \lstinline{scanf} 用于输入,有两个 \lstinline{%d},这个东西被称为“占位符”;要读取几个变量,引号中就要有多少个 \lstinline{%d};
    \item \codex{printf} 用于输出变量,其中 \lstinline{%d} 也是占位符,每个占位符会被依次替换为后面的每个表达式的值。
\end{itemize}

\section{分支选择、数据类型、算术运算}

\begin{lstlisting}[language=C]
    #include <stdio.h>
    #include <math.h> // 引入数学库,使用 sqrt 函数

    int main() {
        double a, b, c; // 定义三个双精度浮点数变量
        printf("Solve the equation ax^2 + bx + c = 0\n");
        printf("Please input the coefficients a, b and c:");
        scanf("%lf%lf%lf", &a, &b, &c); // 读入 a b c
        double delta = b * b - 4 * a * c; // 计算判别式
        if (delta < 0) {
            printf("No real roots.\n");
        } else if (delta == 0) {
            double root = -b / (2 * a);
            printf("One real root: %lf\n", root);
        } else {
            double root1 = (-b + sqrt(delta)) / (2 * a);
            double root2 = (-b - sqrt(delta)) / (2 * a);
            printf("Two real roots: %lf and %lf\n", root1, root2);
        }
        return 0;
    }
\end{lstlisting}

这个例子展示了如何使用分支选择语句 \lstinline{if-else} 来处理不同的情况。
相比于前面的例子,这里你需要了解的内容有:

\begin{itemize}
    \item \lstinline{double} 用于定义双精度浮点数变量,可以存储小数,与 \lstinline{int} 不同;
    \item 在 \lstinline{scanf} 和 \lstinline{printf} 中,使用 \lstinline” 进行加减乘除和取模(即除法求余数)运算;
    \item 使用数学库中的函数,比如这里的 \lstinline{sqrt()} 用于计算平方根;使用这些函数前需要包含相应的头文件:\lstinline{#include <math.h>};其中还有很多常用的数学函数,如 \lstinline{pow()}(幂运算)、\lstinline{sin()}、\lstinline{cos()}(三角函数)等;
    \item 使用 \lstinline{if-else} 语句根据条件执行不同的代码块。
\end{itemize}

\section{Online Judge 练习平台}

Online Judge(OJ)是指在线编程练习平台,提供了大量的编程题目供练习。
常见的 OJ 平台有:洛谷、Codeforces、牛客网、LeetCode 等。

简单介绍一下洛谷平台的使用方法:

\begin{itemize}
    \item 访问洛谷网站:\href{https://www.luogu.com.cn/}{https://www.luogu.com.cn/},注册并登录;
    \item 浏览题库,选择适合自己水平的题目进行练习;
    \item 阅读题目描述,理解题意和输入输出要求;
    \item 在本地编写代码,调试通过后,将代码提交到洛谷平台;
    \item 提交后,系统会自动评测代码的正确性和效率,并给出结果反馈:AC 表示正确,其他结果表示有错误或超时等问题;
    \item 选择提交时要选择 C 语言作为编程语言;
    \item 严格遵守输入输出格式,不要输出多余内容;比如 \ref{sec:input-output} 节中的程序就不适合直接提交到 OJ 平台,因为它包含了提示信息,去除“Please input...”和“The result...”,只输出一个数字表示结果即可。
\end{itemize}

\section{学习方法}

这里能讲到的内容毕竟有限,学习过程中需要善于查找资料,遇到问题时可以:
\begin{itemize}
    \item 查阅 C 语言相关书籍和在线文档,如 \href{https://en.cppreference.com/w/c}{C Reference};这是最靠谱,最权威的参考资料;
    \item 利用搜索引擎查找相关问题的解决方案;
    \item 查阅编程社区和论坛,向其他程序员请教问题;
    \item AI 助手(比如 ChatGPT)也是一个不错的选择,可以用来解答编程相关的问题。
\end{itemize}
擅于自主学习和解决问题是成为优秀程序员的重要能力。

\begin{problemset}[练习]
    \item 在洛谷通过以下题目:
        \href{https://www.luogu.com.cn/problem/P1001}{P1001 A+B Problem}、
        \href{https://www.luogu.com.cn/problem/P1150}{P1150 Peter 的烟}、
        \href{https://www.luogu.com.cn/problem/P1425}{P1425 小鱼的游泳时间}。
    \item 现在你已经掌握了基本的输入、计算、输出三步;接下来请你编写一个程序,读取平面上三个点的坐标,计算并输出它们构成的三角形的三条边的长度、三个角的角度以及三角形面积。
\end{problemset}


\newpage

\textbf{参考答案:}

\begin{lstlisting}[language=C]
    #include <stdio.h>
    #include <math.h>

    int main() {
        const double PI = acos(-1.0);
        double x1, y1, x2, y2, x3, y3;
        printf("Please input the coordinates of three points (x1 y1 x2 y2 x3 y3):");
        scanf("%lf%lf%lf%lf%lf%lf", &x1, &y1, &x2, &y2, &x3, &y3);

        double a = sqrt((x2 - x3) * (x2 - x3) + (y2 - y3) * (y2 - y3));
        double b = sqrt((x1 - x3) * (x1 - x3) + (y1 - y3) * (y1 - y3));
        double c = sqrt((x1 - x2) * (x1 - x2) + (y1 - y2) * (y1 - y2));

        double angleA = acos((b*b + c*c - a*a) / (2*b*c)) * 180.0 / PI;
        double angleB = acos((a*a + c*c - b*b) / (2*a*c)) * 180.0 / PI;
        double angleC = 180.0 - angleA - angleB;

        double s = (a + b + c) / 2;
        double area = sqrt(s * (s - a) * (s - b) * (s - c));

        printf("Sides: a=%.2lf, b=%.2lf, c=%.2lf\n", a, b, c);
        printf("Angles: A=%.2lf°, B=%.2lf°, C=%.2lf°\n", angleA, angleB, angleC);
        printf("Area: %.2lf\n", area);

        return 0;
    }
\end{lstlisting}

\end{document}
